\documentclass[a4paper,12pt]{mwart}

\usepackage{polski}
\usepackage[utf8]{inputenc}
\usepackage{color}
\usepackage{hyperref}

\newcommand{\TODO}[1]{\textcolor{blue}{TODO: #1 \\}}
\newcommand{\ang}[1]{ang.~{\itshape #1}}

\begin{document}

\title{Cebula Deal Finder\\%
{\large czyli System wykrywania okazji sprzedażowych w Internecie} }

\author{Łukasz "Cebula" Jędrzejewski \and Igor "Prohibicja" Rodzik \and Artur "Pieseł" Sawicki}

\maketitle

\section{Opis projektu}
W ramach projektu realizujemy system służący do wykrywania okazji sprzedażowych w internecie. Produktami, jakich ceny porównujemy, są książki.
Program znajduje wpisany przez użytkownika tytuł na paru serwisach internetowych (zaimplementowanych mamy cztery:
\href{http://www.empik.com}{empik.com}, \href{http://www.ksiegarniawarszawa.pl}{ksiegarniawarszawa.pl}, \href{http://www.matras.pl}{matras.pl} i \href{http://www.aros.pl}{aros.pl}
), dopasowuje znalezione pozycje, a następnie porównuje ich ceny i prezentuje je użytkownikowi.

\section{Architektura}
\TODO{Może jakiś diagramik? Ja to nie umiem rysować...}
Do realizacji projektu użyliśmy języka scala oraz frameworka Akka do tworzenia i zarządzania aktorami.
Poniższe sekcje opisują stworzonych przez nas aktorów oraz interakcje między nimi.

\subsection{Władca (\ang{Master})}
Od tego aktora rozpoczyna się interakcja użytkownika z aplikacją. Jego zadaniem jest przyjęcie tytułu do wyszukiwania i przekazanie pracy nowemu koordynatorowi,
a także odbiór wyników pracy od koordynatora i ich prezentacja.

\subsection{Koordynator (\ang{Coordinator})}
Koordynator tworzony jest per wyszukiwany przez użytkownika tytuł. Do jego obowiązków należy przekazanie tytułu do wyszukiwania poszczególnym znajdowaczom,
zebranie wyników od wszystkich znajdowaczy, przekazanie ofert do dopasowania do dopasowywacza i, finalnie, zwrócenie pogrupowanych pozycji do mastera.

\subsection{Znajdowacz (\ang{Finder})}
Każdy z serwisów, na których wyszukujemy książki, ma zaimplementowanego znajdowacza właściwego sobie. Każdy z nich potrafi zebrać ze strony listę wyników wyszukiwania,
a następnie z każdej podstrony zbudować jednolity rezultat oferty. Listę ofert oddaje koordynatorowi. Do pobierania stron każdy ze znajdowaczy korzysta z pobieracza.

\subsection{Pobieracz (\ang{Downloader})}
Zadanie pobieracza to rozdzielanie żądań pobierania między dostępnych workerów, z których każdy pobiera stronę o podanym adresie URL.

\subsection{Dopasowywacz (\ang{Matcher})}
Dopasowywacz potrafi pogrupować otrzymane ze wszystkich serwisów oferty, tak aby móc je porównać i zaprezentować użytkownikowi.
Korzysta w tym celu z algorytmu {\itshape k-medoid}, który zostanie opisany poniżej.

\section{Dopasowywanie ofert}

\subsection{Przygotowanie danych}

\subsection{Algorytm k-medoid}

\subsection{Miara odległości}
\TODO{opis cachowania}

\subsubsection{N-gramy}

\subsubsection{Miara kosinusowa i~TFIDF}

\subsection{Wybór liczby grup}

W~zagadnieniu grupowania duży problem stanowi odpowiedni wybór liczby grup.
Czasem zdarza się, że~wiemy z~góry ile ich powinno być. Jednak w~naszym
projekcie stanęliśmy przed koniecznością wyboru liczby grup automatycznie.

Pierwszym rozwiązaniem, które opracowaliśmy było uruchomienie algorytmu
grupowania danych dla zakresu grup $[2, \lceil \frac{n}{2} \rceil]$, gdzie $n$
to~liczba przykładów. Po~wyznaczeniu grup, sprawdzaliśmy sumy średnich
odległości między przykładami w~klastrach. Im~opisana suma była mniejsza, tym
grupowanie traktowaliśmy za~lepsze. Mankamentem tego sposobu oceny jest fakt,
że~każde grupowanie dla większej liczby grup, będzie lepsze. W~szczególności
dla grupowania, w~którym liczba przykładów pokrywa się z~liczbą grup, miara
oceny wyniesie $0$.

W~związku z~tym postanowiliśmy skorzystać z~jednej dostępnych metod oceny
jakości grupowania, która nie wymaga wiedzy o~\textbf{prawdziwych}
przypisaniach do~grup. Zaimplementowaliśmy miarę sylwetki (\ang{silhouette}).
Jej idea polega na~skonfrontowaniu średniej niepodobieństw do~przykładów
z~takiego samego klastra, jak rozważany przykład -- $a(i)$, gdzie $i$
to~przykład, oraz najmniejszej średniej niepodobieństw przykładów należących
do~pozostałych klastrów -- $b(i)$. Następnie zdefiniowany jest wskaźnik:

\[s(i) = \frac{b(i) - a(i)}{\max[a(i), b(i)]}\]

którego najbardziej pożądana wartość powinna być bliska $1$, co~oznacza,
że~przykład jest znacznie bliższy przykładom w~grupie, do~której należy, niż
do~jakiejkolwiek z~innych grup. Taka sytuacja zachodzi dla $a(i) \ll~b(i)$.

Wartość miary sylwetki należy do~przedziału $[-1, 1]$, co~kłóciło się
z~przyjętym w~projekcie założeniem, że~im~mniejsza wartość tym lepsze
grupowania. Dlatego zmodyfikowaliśmy miarę następująco:

\[s'(i) = \left | s(i) - 1 \right | \]

Ostatecznie zależało nam na~wskaźniku dla całego grupowania, dlatego powyższą
wartość uśredniamy na~wszystkich przykładach.

Po~zastosowaniu miary sylwetki do~wyboru najlepszego efekty nie były dużo
lepsze od~pierwszego sposobu oceny. Co~prawda dla niewielkiej liczby grup,
wartość miary była dość zróżnicowana, to~jednak wraz ze~wzrostem liczby
klastrów, ciągle malała.

Ta~obserwacja skłoniła nas do~dodania wpływu wielkości grupy na~miarę.
Zmieniliśmy tylko formułę $a(i)$, czyli średnią odległość przykładu do~innych
przykładów w~jego grupie, w~poniższy sposób:

\begin{itemize}
\item grupy jednoelementowe karzemy -- zwracamy $0{,}1$, zamiast $0$
  w~oryginalnej mierze,
\item z~kolei większe grupy zdecydowaliśmy się promować -- zwracamy $a(i) /
  \left (1 + \log_{10} \left | C(i) \right | \right)$, gdzie $\left | C(i)
  \right |$ to~liczba przykładów w~grupie przykładu $i$, dzieląc przez czynnik
  zależny od~liczności grupy.
\end{itemize}

Dopiero tak zmodyfikowana miara sylwetki przyniosła pewną poprawę.

\section{Wyniki}
\TODO{A jakie wyniki tu wrzucamy?}

\section{Podsumowanie}
\TODO{A tu to już wogóle nie wiem co napisać...}

\end{document}
