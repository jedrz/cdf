\documentclass{article}

\usepackage{polski}
\usepackage[utf8]{inputenc}
\usepackage{color}
\usepackage{hyperref}

\newcommand{\TODO}[1]{\textcolor{blue}{TODO: #1 \\}}
\newcommand{\ang}[1]{ang.~{\itshape #1}}

\begin{document}

\title{Cebula Deal Finder\\%
{\large czyli System wykrywania okazji sprzedażowych w Internecie} }

\author{Łukasz "Cebula" Jędrzejewski \and Igor "Prohibicja" Rodzik \and Artur "Pieseł" Sawicki}

\maketitle

\section{Opis projektu}
W ramach projektu realizujemy system służący do wykrywania okazji sprzedażowych w internecie. Produktami, jakich ceny porównujemy, są książki.
Program znajduje wpisany przez użytkownika tytuł na paru serwisach internetowych (zaimplementowanych mamy cztery:
\href{http://www.empik.com}{empik.com}, \href{http://www.ksiegarniawarszawa.pl}{ksiegarniawarszawa.pl}, \href{http://www.matras.pl}{matras.pl} i \href{http://www.aros.pl}{aros.pl}
), dopasowuje znalezione pozycje, a następnie porównuje ich ceny i prezentuje je użytkownikowi.

\section{Architektura}
\TODO{Może jakiś diagramik? Ja to nie umiem rysować...}
Do realizacji projektu użyliśmy języka scala oraz frameworka Akka do tworzenia i zarządzania aktorami.
Poniższe sekcje opisują stworzonych przez nas aktorów oraz interakcje między nimi.

\subsection{Władca (\ang{Master})}
Od tego aktora rozpoczyna się interakcja użytkownika z aplikacją. Jego zadaniem jest przyjęcie tytułu do wyszukiwania i przekazanie pracy nowemu koordynatorowi,
a także odbiór wyników pracy od koordynatora i ich prezentacja.

\subsection{Koordynator (\ang{Coordinator})}
Koordynator tworzony jest per wyszukiwany przez użytkownika tytuł. Do jego obowiązków należy przekazanie tytułu do wyszukiwania poszczególnym znajdowaczom,
zebranie wyników od wszystkich znajdowaczy, przekazanie ofert do dopasowania do dopasowywacza i, finalnie, zwrócenie pogrupowanych pozycji do mastera.

\subsection{Znajdowacz (\ang{Finder})}
Każdy z serwisów, na których wyszukujemy książki, ma zaimplementowanego znajdowacza właściwego sobie. Każdy z nich potrafi zebrać ze strony listę wyników wyszukiwania,
a następnie z każdej podstrony zbudować jednolity rezultat oferty. Listę ofert oddaje koordynatorowi. Do pobierania stron każdy ze znajdowaczy korzysta z pobieracza.

\subsection{Pobieracz (\ang{Downloader})}
Zadanie pobieracza to rozdzielanie żądań pobierania między dostępnych workerów, z których każdy pobiera stronę o podanym adresie URL.

\subsection{Dopasowywacz (\ang{Matcher})}
Dopasowywacz potrafi pogrupować otrzymane ze wszystkich serwisów oferty, tak aby móc je porównać i zaprezentować użytkownikowi.
Korzysta w tym celu z algorytmu {\itshape k-medoid}, który zostanie opisany poniżej.

\section{Algorytm {\itshape k-medoid}}
\TODO{Piszemy coś o tym wogóle?}
\TODO{A o cachowaniu coś piszemy?}

\section{Wyniki}
\TODO{A jakie wyniki tu wrzucamy?}

\section{Podsumowanie}
\TODO{A tu to już wogóle nie wiem co napisać...}

\end{document}
